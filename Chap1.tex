
\chapter{Standard model}\label{chap:one}

\section{Introduction to quantum field theory}\label{intro}
Twentieth century begins with new developments to every branch the physics. Specifically, theories such as special relativity started probing the realm of objects that are moving with velocities comparable to the vacuum speed of light. Whereas, theories such as quantum mechanics seem to describe the phenomena that arise in smaller objects moving with nonrelativistic velocities. However, it appears that these two major theories could be merged to describe the motion of smaller particles with relativistic velocities. This eventually produced the concept of \textit{relativistic quantum mechanics}. Although relativistic quantum mechanics was a good approximation to the phenomena that were observed at the relativistic regime, it was not consistent with the observation. It was apparent that particles were not the fundamental interpretation, as well. Quantum fields seem to describe the dynamics of particles and their interactions more consistently. The field is a mathematical construction that is present at every point in space-time. According to the quantum field description, a particle is considered as an excitation (ripple) in the quantum field.  Therefore, the interactions between these quantum fields describe the interaction between particles. This quantum field interpretation, however, was the birth of a new era of particle physics. Currently, \textit{quantum filed theory} is the most fundamental interpretation for relativistic particle interaction.\\
Write about CPV and B to X gamma!
\section{Standard Model (SM)}
The \textit{standard model} of particle physics describes three fundamental interactions in nature using the quantum field theory. Besides, the SM is a renormalizable theory. Thus, all the interactions can be decried by the finite number of parameters such as mass, coupling constants, and wave functions.\\
The \textit{quantum electrodynamics} describe the electromagnetic interactions, the weak nuclear interactions explained the spontaneous decay of the elementary particles and \textit{quantum chromodynamics } describe the strong nuclear interactions between particles. The Glashow-Weinberg-Salam (GWS) theory \cite{Glashow:1961tr, Weinberg:1967tq} pointed out that electromagnetic interactions and weak interactions can be described by a single theory, which is known as \textit{electro weak theory}.\\
According to the SM point of view, there are matter fields and force mediating fields. The matter fields are associated with intrinsic spin $1/2$, and they are known as \textit{fermions}. Whereas force-carrying fields are associated with integral spins. Therefore, these force-carrying fields are also known as \textit{gauge bosons}. The quantization of the scalar field yields a particle with $0$ spin. Similarly \textit{vector} fields provides spin 1 particle.\\  

\subsection{Symmetry}\label{symmetry}
In particle physics, the fields are defined using the complex-valued mathematical spaces. For an example, electromagnetism is set using complex numbers. Also, it satisfies the symmetry called $U(1)$. This symmetry is an abstract internal symmetry of the electromagnetic field that is realized in complex space. Similarly, the massless week boson field satisfies another abstract symmetry called $SU(2)$. These symmetries are based on the rotations of two and three-dimensional spaces.\\
As the Young-Mills theory \cite{Yang:1954ek} points out, the associated Lagrangian for these interacting fields should be invariant under local gauge transformations. This means the Lagrangian remains the same after an internal rotations. However, this theory seems to work well only for mass-less gauge fields such as photon field and gluon field. For the heavy vector bosons such as $W$ and $Z$ requires another mechanism to generate their masses. For this a scalar field called \textit{higgs} field is employed.\\
Higgs field is described by a quadratic potential. Unlike the other fields, the lowest energy state of the higgs field is not its vacuum state (a state free of excitation). Therefore, it starts decaying from free of excitation (vacuum) state into a low energy state with a field value. This field value is known as vacuum expectation value ($\nu$). The interaction between this $\nu$ and the particle fields is represented by a mass term to the originally mass less particle fields in the Lagrangian. \\
In summery the standard model is specified by complex group $SU(3)_c\times SU(2)_L\times U(1)_Y$. The quantum chromodynamics (QCD) has non-abelian gauge symmetry called $SU(3)_C$, the weak and electromagnetic interactions exhibits $SU(2)_L\times U(1)_Y$ symmetry.  As stipulated by the higgs mechanism, the $SU(2)_L\times U(1)_Y$ breaks to electromagnetic subgroup $U(1)_Q$. The coupling for strong, weak and electromagnetic interactions are given by $g_s$ for strong interaction, $g'$ for weak hyperacharge $U(1)_Y$ and $g$ for weak isospin $SU(2)_L$. The generator of week hypercharge is denoted by $Y$, and the three generators of week isospin is given by $\tau^i$ where $i=1,2,3$. For the $SU(3)_C$ there are 8 generators, which are denoted by $T^a$, where $a=1,...,8$. Since the interactions between these fields have a complex structure, it is easier to divide the SM Lagrangian into several sectors for the following analysis.

\subsection{The gauge sector}
The gauge field for strong interactions are given by $G_{\mu}^a$, for the $SU(2)_L$ the gauge field is given by $W_{\mu}^i$ and the gauge field for the $U(1)_Y$ is given by $B_{\mu}$. The corresponding abelian/non-abelian field strength tensors for these gauge fields are given as follows:
\begin{eqnarray}\label{GaugeFieldStreangth}
\begin{aligned} B_{\mu \nu} &=\partial_{\mu} B_{\nu}-\partial_{\nu} B_{\mu} \\ W_{\mu \nu}^{i} &=\partial_{\mu} W_{\nu}^{i}-\partial_{\nu} W_{\mu}^{i}-g \epsilon^{i j k} W_{\mu}^{j} W_{\nu}^{k} \\ G_{\mu \nu}^{a} &=\partial_{\mu} G_{\nu}^{a}-\partial_{\nu} G_{\mu}^{a}-g_{s} f^{a b c} G_{\mu}^{b} G_{\nu}^{c} \end{aligned}
\end{eqnarray}
Therefore, in the SM Lagrangian gauge sector is realized as follows:
\begin{eqnarray}
\mathcal{L}_{\mathrm{gauge}}=-\frac{1}{4} B_{\mu \nu} B^{\mu \nu}-\frac{1}{4} W_{\mu \nu}^{i} W^{i, \mu \nu}-\frac{1}{4} G_{\mu \nu}^{a} G^{a, \mu \nu}
\end{eqnarray}
\subsection{The fermionic sector}
The fermionic sector contains the matter fields. Also, it exhibits the $SU(2)_L\times U(1)_Y$ symmetry, which accounts the weak and electromagnetic  interactions. There are three generations of fermions each consist of \textit{neutrino} $v_i$, \textit{lepton} with electromagnetic charge $Q_i=-1$, up type \textit{quark} with $Q_i=+2/3$ and down type quark with $Q_i=-1/3$. $SU(2)_L$ determines the transformation properties of these fermion fields under weak charge. The left handed fermions can mix within each generation. Therefore, these fields should arrange as a $2\times 1$ column vector. This is known as \textbf{2} representation. For an example, $u_L$ and $d_L$ together form \textbf{2} representation of $SU(2)_L$. Similarly, $v_{eL}$ and $e_L$ also transform together to form doublet. However, right handed fields transform as singlets under $SU(2)_L$.\\
The representation of $U(1)_Y$ is the \textit{hypercharge} of the field. The hypercharge is assigned based on the final electromagnetic charge of the fermion. The representation of $SU(3)_C$ is determined by the color charge. The left and right handed quarks comes in three colors this makes them to categorized into a $3\times 1$ column vector. This is the \textbf{3} representation in $SU(3)_C$. Whereas, leptons do not carry a color. Therefore, they are categorized into the singlet representation of $SU(3)_C$.
Altogether, for an example, the transformation of up type quark under $SU(3)_c\times SU(2)_L\times U(1)_Y$ can be shown as follows:
\begin{eqnarray}
u_L\sim (3,2,\frac{1}{3})
\end{eqnarray}
The interaction between matter and the gauge fields are determined by the covariant derivative.
\begin{eqnarray}
D_{\mu}=\partial_{\mu}+i g^{\prime} B_{\mu} Y+i g W_{\mu}^{i} \tau^{i}+i g_{s} G_{\mu}^{a} T^{a}
\end{eqnarray} 
Using this the fermionic part of the standard model Langrangian can be written as follows:
\begin{eqnarray}
\mathcal{L}_{\text{fermionic}}=\sum_{i=1}^{3}\left(\bar{E}^i_L i \slashed{D}E_{L}^i+\bar{Q}_{L}^i i\slashed{D}Q_L^i+\bar{e}_R^i i\slashed{D}e_{R}^i+\bar{u}_R^i i\slashed{D}u_R^i+\bar{d}_R^i i\slashed{D}d_R^i\right)
\end{eqnarray}
where 
\begin{eqnarray}
\begin{aligned} E_{L}^{i} &=P_{L}\left(\begin{array}{c}{\nu_{i}} \\ {e_{i}}\end{array}\right)=\left(\left(\begin{array}{c}{\nu_{e}} \\ {e}\end{array}\right)_{L},\left(\begin{array}{c}{\nu_{\mu}} \\ {\mu}\end{array}\right)_{L},\left(\begin{array}{c}{\nu_{\tau}} \\ {\tau}\end{array}\right)_{L}\right) \\ Q_{L} &=P_{L}\left(\begin{array}{c}{u_{i}} \\ {d_{i}}\end{array}\right)=\left(\left(\begin{array}{c}{u} \\ {d}\end{array}\right)_{L},\left(\begin{array}{c}{c} \\ {s}\end{array}\right)_{L},\left(\begin{array}{c}{t} \\ {b}\end{array}\right)_{L}\right), \end{aligned}
\end{eqnarray}
$P_{L / R}=\left(1 \mp \gamma^{5}\right) / 2$.\\
%Lagrangian density can be devided into couple of sections as follows.
\subsection{Higgs sector}\label{higgssec}
As discussed in the section \ref{symmetry}, the higgs field is introduced as mass generating mechanism to heavy vector bosons. For this a Lagrangian, which is invariant under $O(4)$ rotations, is defined as follows \cite{Manohar:2000dt, Lancaster:2014pza, LlewellynSmith:1973yud}: 
\begin{eqnarray}\label{higgsLagrangian}
\mathcal{L}_{\mathrm{H}}=\left(D_{\mu} \phi\right)^{\dagger}\left(D^{\mu} \phi\right)+\frac{1}{2}\mu^{2} \phi^{\dagger} \phi-\frac{1}{4}\lambda\left(\phi^{\dagger} \phi\right)^{2}+\mathcal{L}_{gauge}
\end{eqnarray}
where 
\begin{eqnarray}
\langle\phi\rangle=\left(\begin{array}{l}{\phi^+} \\ {\phi^0}\end{array}\right)=\frac{1}{\sqrt{2}}\left(\begin{array}{l}{\phi_3+i\phi_4} \\ {\phi_1+i\phi_2}\end{array}\right)
\end{eqnarray}
Here $\mu$ and $\lambda$ are both real parameters. However, the minimum of the higgs potential is not at $\langle\phi\rangle=0$. The \textit{Taylor} expansion of the potential term provides the spherical shell of minima at a radius $\nu=\left(\frac{\mu^2}{\lambda}\right)^{\frac{1}{2}}$. On the surface of this spherical shell there are infinitely many equivalent vacua. The higgs field spontaneously pick one of these vacua and breaks the symmetry. For simplicity, consider the following vacuum field configuration in 3 dimensional isospin space:
\begin{eqnarray}
(\phi_1^0)^2=\frac{2\mu^2}{\lambda}=2\nu^2, \text{  } \phi_2^0=0\text{  }\phi_3^0=0\text{  }\phi_4^0=0
\end{eqnarray}
More concisely, 
\begin{eqnarray}\label{higgsvcuum}
\langle\phi\rangle=\left(\begin{array}{l}{0} \\ {v}\end{array}\right)
\end{eqnarray}
The symmetry breaking $SU(3)_L\times U(1)_Y\rightarrow U(1)_Q$ does not break all the symmetries. For an example, our choice of vacuum given in equation (\ref{higgsvcuum}) is still invariant under $\hat{U}=e^{i(\frac{Y}{2}+I_3\tau^3)\alpha(x)}$ transformation. In fact $Q=\frac{Y}{2}+I_3$ where $Q$ is the electromagnetic charge. Using this the above transformation can be written as  $\hat{U}=e^{iQ\alpha(x)}$. According to the \textit{Yang-Mills} theory this is equivalent to U(1) transformation. Therefore, the electromagnetic interaction emerges from this symmetry breaking. \\
The excitation above the vacuum state of the higgs field is given below:
\begin{eqnarray}
\phi(x)=\left(\begin{array}{l}{0} \\ {v+\frac{h(x)}{\sqrt{2}}}\end{array}\right)
\end{eqnarray}
Also, this, however, redefines the first term in the higgs Lagrangian
\begin{eqnarray}
(D_{\mu}^{\dagger}\phi)(D^{\mu}\phi)=\frac{1}{2}(\partial_{\mu}h(x))^2+\frac{g^2\nu^2}{4}(W_{\mu}^1)^2+\frac{g^2\nu^2}{4}(W^2_{\mu})^2+\frac{\nu^2}{4}(g W_{\mu}^3-g'B_{\mu})^2+...
\end{eqnarray}
Note that the mass of a quantum field has the form $\frac{1}{2}(mass)^2\times(field)^2$. Following from this, the $W_{\mu}^1$ and $W_{\mu}^2$ fields seems to develop a mass $M^2_W=\frac{g^2\nu^2}{2}$ from the interaction. The linear combination of the fields $(g W_{\mu}^3-g'B_{\mu})$ also grown massive. The higgs field components $\phi_2,\phi_3$ and $\phi_4$ disappeared from the interaction Lagrangian. This is known as the mass less gauge fields \textit{eating} the \textit{Goldstone} modes. The massive excitation of the scalar field is known as \textit{higgs} boson.\\
The linear combination $g'W^3_{\mu}+gB_{\mu}$ does not appear in the above Lagrangian. Therefore, this combination is assigned to the mass-less photon field. The \textit{Weinberg angle} is defined using the ratio of coupling constants $g$ and $g'$ as $tan \theta_W=\frac{g'}{g}$. Using this angle two new fields can be defined as follows:
\begin{eqnarray}
\left(\begin{array}{l}{Z_{\mu}} \\ {A_{\mu}}\end{array}\right)=\begin{pmatrix} 
cos\theta_W & -sin\theta_W \\
sin\theta_W & cos\theta_W 
\end{pmatrix}\left(\begin{array}{l}{W_{\mu}^3} \\ {B_{\mu}}\end{array}\right)
\end{eqnarray}  
where $Z_{\mu}$ is the $Z$ boson field and $A_{\mu}$ is the photon field.
\subsection{Yukawa sector}
The higgs couplings between the fermions in SM is described by the \textit{Yukawa} Lagrangian. However, the Lagrangian needs to be \textit{Lorentz invariant} and mass dimension 4. Therefore, the most generalized form can be obtained as follows:
\begin{eqnarray}
\mathcal{L}_{\mathrm{Yukawa}} \supset -y_{\psi}\bar{\psi}\phi\psi+ c.c
\end{eqnarray} 
where $\psi$ and $\phi$ are fermion field and scalar field respectively. Furthermore, the $\phi$ is $SU(2)_L$ doublet. This makes $\psi$ also a $SU(2)_L$ doublet and $\bar{\psi}$ a $SU(2)_L$ singlet to preserve the gauge symmetry. $y_{\psi}$ is the dimensionless coupling between the scalar and fermion fields, which is known as \textit{yukawa coupling}.
\subsubsection{Lepton sector}
Using the above definition the yukawa term in the Lagrangian can be obtained for  $SU(2)_L$ doublet $L_{L}=\left(\nu_{L}, e_{L}\right)^{T}$ as follows:
\begin{eqnarray}
\mathcal{L}_{\mathrm{Yukawa}} \supset-\left[y_{e} \overline{e}_{R} \Phi^{\dagger} L_{L}+y_{e}^{*} \overline{L}_{L} \Phi e_{R}\right]
\end{eqnarray}
The coupling $y_{e}$ is obtained using
\begin{eqnarray}
\frac{y_{e}}{\sqrt{2}}=\frac{m_{e}}{v}=\frac{511 \mathrm{keV}}{246 \mathrm{GeV}} \simeq 2.1 \times 10^{-6}
\end{eqnarray} 	
Also, this can be extended to all three generations in SM. This gives a generalized lepton sector as
\begin{eqnarray}
\mathcal{L}_{\mathrm{lepton}} \supset-\left[Y_{ij} \overline{E}_{Rj} \Phi^{\dagger} L_{Li}+Y_{ij}^{*} \overline{L}_{Li} \Phi E_{Rj}\right]
\end{eqnarray} 
Note that in SM $neutrino$ does not couples to higgs field and does not generate mass. This is given by absence of the $\nu_{Ri}$ fields. $Y_{ij}$ is the matrix element of yukawa matrix.
\subsubsection{Quark sector}
Similarly the yukawa interaction between $SU(2)_L$ quark doublet $Q_L=\left(u_{L}, d_{L}\right)^{T}$ and down type quark singlet $d_R$ can be written as 
\begin{eqnarray}\label{downType}
\mathcal{L}_{\mathrm{Yukawa}} \supset-\left[y_{d} \overline{d}_{R} \Phi^{\dagger} Q_{L}+y_{d}^{*} \overline{Q}_{L} \Phi d_{R}\right]
\end{eqnarray}
However, the mass generation of up type quarks needs an additional higgs doublet, which is denoted by $\tilde{\phi}$. This doublet is obtained by using the conjugate doublet transformation in $SU(2)$. Therefore, the conjugate higgs doublet is given by
\begin{eqnarray}
\tilde{\phi} \equiv i \sigma^{2} \phi^{*}=i\left(\begin{array}{cc}{0} & {-i} \\ {i} & {0}\end{array}\right)\left(\begin{array}{c}{\phi^{-}} \\ {\phi^{0 *}}\end{array}\right)=\left(\begin{array}{c}{\phi^{0 *}} \\ {-\phi^{-}}\end{array}\right)
\end{eqnarray}
Also, as an artifact of this $\tilde{\phi}$ obtains $Y=-\frac{1}{2}$. Using this doublet another gauge invariant term can be written as follows:
\begin{eqnarray}\label{upType}
\mathcal{L}_{\mathrm{Yukawa}} \supset-\left[y_{u} \overline{u}_{R} \tilde{\phi}^{\dagger} Q_{L}+y_{u}^{*} \overline{Q}_{L} \tilde{\phi} u_{R}\right]
\end{eqnarray}
As shown before the quark masses and the higgs vev defines the yukawa couplings.  This can also be scaled to all 3 generations of quarks as well. Therefore, the by counting the lepton sector we arrive at most generalized yukawa term,
 \begin{eqnarray}\label{SMYukawa}
 -\mathcal{L}_{\mathrm{Yukawa}}=Y_{i j}^{d} \overline{Q}_{L i} \phi D_{R j}+Y_{i j}^{u} \overline{Q}_{L i} \tilde{\phi} U_{R j}+Y_{i j}^{e} \overline{L}_{L i} \phi E_{R j}+\mathrm{h.c.}
 \end{eqnarray}
Also, this term is the source of all \textit{flvor} interactions \cite{Isidori:2010kg}. 
\subsection{The Standard model Lagrangian}
Considering all the interaction between gauge bossons, fermions and scalars the final Lagrangian that describes the SM can be written as follows: 
\begin{eqnarray}\label{SMLagarangian}
\mathcal{L}_{\mathrm{SM}}=\mathcal{L}_{\text { fermionic + gauge }}+\mathcal{L}_{\text { Higgs }}+\mathcal{L}_{\text { Yukawa }}
\end{eqnarray}
\begin{center}
\begin{tabular}{ |c|c|c|c| } 
\hline
Type\quad spin & Field & Multiplet \\ 
\hline
\multirow{3}{4em}{Vector\, 1} & $B_{\mu}$ & (1,1,0) \\ 
& $W_{\mu}$ & (1,3,0) \\ 
& $G_{\mu}$& (8,1,0) \\ 
\hline
\multirow{5}{4em}{Spinor\, $\frac{1}{2}$} & $E_{Li}$ & (1,2,$\frac{-1}{2}$) \\ 
& $Q_{Li}$ & (3,2,$\frac{1}{6}$) \\ 
& $e_{Ri}$& (1,1,-1) \\ 
& $u_{Ri}$ & (3,1,$\frac{2}{3}$) \\
& $d_{Ri}$ & (3,1,$\frac{-1}{3}$) \\
\hline
Sacalar\quad 0 & $\phi$ & (1,2,$\frac{1}{2}$)\\
\hline
\end{tabular}
\end{center}
\subsection{Cabibbo-Kobayashi-Maskawa (CKM) matrix}  
Consider the equations (\ref{downType}) and (\ref{upType}). They can be generalized to all three generations of quarks in the SM.
\begin{eqnarray}
\mathcal{L}_{\mathrm{Yukawa}}^{\text{quark}}=-\sum_{i=1}^{3} \sum_{j=1}^{3}\left[y_{i j}^{u} \bar{u}_{R i} \tilde{\phi}^{\dagger} Q_{L j}+y_{i j}^{d} \bar{d}_{R i} \phi^{\dagger} Q_{L j}\right]+\mathrm{h.c.}
\end{eqnarray}
The dimensionless \textit{yukawa} couplings now become $3\times3$ matrices. These matrices contain 18 complex parameters. As shown in the section [higgs section], replacing the higgs by it's vacuum configuration $\phi=(0, \nu/\sqrt{2})^T$ provides the mass term. 
\begin{eqnarray}
\mathcal{L}_{\mathrm{Yukawa}}^{q} \supset-\left(\bar{u}_{1}, \bar{u}_{2}, \bar{u}_{3}\right)_{R} \mathcal{M}^{u}\left(\begin{array}{c}{u_{1}} \\ {u_{2}} \\ {u_{3}}\end{array}\right)_{L}-\left(\bar{d}_{1}, \bar{d}_{2}, \bar{d}_{3}\right)_{R} \mathcal{M}^{d}\left(\begin{array}{c}{d_{1}} \\ {d_{2}} \\ {d_{3}}\end{array}\right)_{L}+\mathrm{h.c.},
\end{eqnarray}
where 
\begin{eqnarray}
\mathcal{M}_{i j}^{u}=\frac{v}{\sqrt{2}} y_{i j}^{u}, \quad \mathcal{M}_{i j}^{d}=\frac{v}{\sqrt{2}} y_{i j}^{d}.
\end{eqnarray}\
The $\mathcal{M}_{i j}^{u}$ and $\mathcal{M}_{i j}^{d}$ are kmnown as the quark mass matrices in the \textit{generation} space. The diagonalization of these mass matrices provide the quark mass eiganstates. This is done by multiplying the mass matrices by unitary matrices $U_{L}, U_{R}, D_{L}$ and $D_{R}$. They are defined by
\begin{eqnarray}
\left(\begin{array}{c}{u_{1}} \\ {u_{2}} \\ {u_{3}}\end{array}\right)_{L, R}=U_{L, R}\left(\begin{array}{c}{u} \\ {c} \\ {t}\end{array}\right)_{L, R}, \quad\left(\begin{array}{c}{d_{1}} \\ {d_{2}} \\ {d_{3}}\end{array}\right)_{L, R}=D_{L, R}\left(\begin{array}{c}{d} \\ {s} \\ {b}\end{array}\right)_{L, R},
\end{eqnarray}
where $u,c,t,d,s$ and $b$ are quark mass eigenstates. This gives us the diagonalized mass matrices. 
\begin{eqnarray}
U_{R}^{-1} \mathcal{M}^{u} U_{L}=\left(\begin{array}{ccc}{m_{u}} & {0} & {0} \\ {0} & {m_{c}} & {0} \\ {0} & {0} & {m_{t}}\end{array}\right), \quad D_{R}^{-1} \mathcal{M}^{d} D_{L}=\left(\begin{array}{ccc}{m_{d}} & {0} & {0} \\ {0} & {m_{s}} & {0} \\ {0} & {0} & {m_{b}}\end{array}\right).
\end{eqnarray}
Also, $\mathcal{M}^u$ and $\mathcal{M}^d$ diagonalizes \textit{yukawa} matrices $y_{i j}^{u}=\frac{\sqrt{2}}{v} \mathcal{M}_{i j}^{u} \text { and } y_{i j}^{d}=\frac{\sqrt{2}}{v} \mathcal{M}_{i j}^{d}$.\par
These mass eigenstates (physical states) of up and down type quarks can be coupled in charged current interactions.
\begin{eqnarray}
J_{L}^{+\mu}=\left(\bar{u}_{1}, \bar{u}_{2}, \bar{u}_{3}\right)_{L} \gamma^{\mu}\left(\begin{array}{c}{d_{1}} \\ {d_{2}} \\ {d_{3}}\end{array}\right)=(\bar{u}, \bar{c}, \bar{t})_{L} U_{L}^{\dagger} \gamma^{\mu} D_{L}\left(\begin{array}{c}{d} \\ {s} \\ {b}\end{array}\right)_{L}=(\bar{u}, \bar{c}, \bar{t})_{L} \gamma^{\mu} V\left(\begin{array}{c}{d} \\ {s} \\ {b}\end{array}\right)_{L}.
\end{eqnarray} 
Here $V=U_{L}^{\dagger}D_{L}$ is known as the \textit{Cabibbo-Kobayashi-Maskawa} (CKM) matrix, and  it is given by 
\begin{eqnarray}
V_{\text{CKM}}=\left(\begin{array}{ccc}{V_{u d}} & {V_{u s}} & {V_{u b}} \\ {V_{c d}} & {V_{c s}} & {V_{c b}} \\ {V_{t d}} & {V_{t s}} & {V_{t b}}\end{array}\right)
\end{eqnarray}
The CKM is unitary 
\begin{eqnarray}
V^{\dagger} V=\left(U_{L}^{\dagger} D_{L}\right)^{\dagger}\left(U_{L}^{\dagger} D_{L}\right)=D_{L}^{\dagger} U_{L} U_{L}^{\dagger} D_{L}=1
\end{eqnarray}
Since CKM is a $3\times 3$ matrix its defined by 9 complex parameters (18 real numbers). The constraint $V_{a b}^{\dagger} V_{b c}=\delta_{a c}$ reduce this to 9 real parameters. The redefinition $q_{L} \rightarrow e^{i\alpha_{q_L}} q_{L}$ can technically remove 6 phases because there are 6 different quark fields. However, the common phase redefinition of all the quarks does not affect $V$. This, in turn, reduces the number of nonphysical phases to 5. Altogether, there are 4 independent parameters to describe the CKM matrix. There is no unique parameterization for the CKM matrices. The most common ones are `` Standard parametarization" \cite{Chau:1984fp} and ``Wolfenstein parameterization"\cite{Wolfenstein:1983yz}. 
\subsubsection{Standard Parameterization }
The Standard parameterization is given by
\begin{eqnarray}
V_{\text{CKM}}=\left(\begin{array}{ccc}{c_{12} c_{13}} & {s_{12} c_{13}} & {s_{13} e^{-i \delta}} \\ {-s_{12} c_{23}-c_{12} s_{23} s_{13} e^{i \delta}} & {c_{12} c_{23}-s_{12} s_{23} s_{13} e^{i \delta}} & {s_{23} c_{13}} \\ {s_{12} s_{23}-c_{12} c_{23} s_{13} e^{i \delta}} & {-s_{23} c_{12}-s_{12} c_{23} s_{13} e^{i \delta}} & {c_{23} c_{13}}\end{array}\right),
\end{eqnarray}
where $s_{ij}=sin\theta_{ij}$ and $c_{ij}=cos\theta_{ij}(i=1,2,3)$. The phase $\delta$ is necessary for the CP violation, and it's range $0\leq \delta\leq 2\pi$. The measurements of CPV in $K$ decays constrain this range to $0\leq \delta\leq \pi$ \cite{Buras:1998raa}.\par
The $s_{13}$ and $s_{23}$ are in the order of $10^{-3}$, and $c_{13}=c_{23}=1$ \cite{Tanabashi:2018oca}. This leaves 4 independent parameters 
\begin{eqnarray}
s_{12}=\left|V_{u s}\right|, \quad s_{13}=\left|V_{u b}\right|, \quad s_{23}=\left|V_{c b}\right|, \quad \delta
\end{eqnarray}
\subsubsection{Wolfensteine Parameterization}
Wolfensteine Parameterization can be obtained by defining the independent parameters in standard parameterization by $\lambda, A, \rho, \eta$ \cite{Buras:1994ec}.
\begin{eqnarray}
s_{12}=\lambda, \quad s_{23}=A \lambda^{2}, \quad s_{13} e^{-i \delta}=A \lambda^{3}(\varrho-i \eta)
\end{eqnarray}
This is an approximate parameterization, in which each CKM elements is expanded in power series of small parameter $\lambda = |V_{us}|=0.22$. This gives us 
\begin{eqnarray}
\hat{V}=\left(\begin{array}{ccc}{1-\frac{\lambda^{2}}{2}} & {\lambda} & {A \lambda^{3}(\varrho-i \eta)} \\ {-\lambda} & {1-\frac{\lambda^{2}}{2}} & {A \lambda^{2}} \\ {A \lambda^{3}(1-\varrho-i \eta)} & {-A \lambda^{2}} & {1}\end{array}\right)+\mathcal{O}\left(\lambda^{4}\right).
\end{eqnarray}
\subsubsection{Unitarity triangles}
Unitarity relationships are obtained by triangle relations defined on complex plane. For example, 
\begin{eqnarray}
\begin{array}{l}{V_{u d} V_{u b}^{*}+V_{c d} V_{c b}^{*}+V_{t d} V_{t b}^{*}=0} \\ {V_{u  s} V_{u  b}^{*}+V_{c s} V_{c b}^{*}+V_{t s} V_{t b}^{*}=0} \\ {V_{u d } V_{u s}^{*}+V_{c d} V_{c s}^{*}+V_{t d} V_{t s}^{*}=0} \\ {V_{u d} V_{t d}^{*}+V_{u s} V_{t s}^{*}+V_{u b} V_{t b}^{*}=0} \\ {V_{c d} V_{t d}^{*}+V_{c s} V_{t s}^{*}+V_{c b} V_{t b}^{*}=0} \\ {V_{u d } V_{c d}^{*}+V_{u s} V_{c s}^{*}+V_{u b} V_{c b}^{*}=0}\end{array}
\end{eqnarray}
The area of the unitarity triangles provide the measurement of CP violation ($J_{CP}$). This measurement is obtained by\cite{Jarlskog:1988ii} 
\begin{eqnarray}
\left|J_{\mathrm{CP}}\right|=2 \cdot A_{\Delta}
\end{eqnarray}
where $A_{\Delta}$ is the area of the unitarity triangle. Therefore, precise measurements on CKM parameters along with these unitarity relationships gives us important information on CP violation. 
\subsection{Flavor physics}
In \textit{flavor physics} the interactions between different flavors are studied extensively. Mass less gauge bosons such as \textit{gluons} and \textit{photons} do not distinguish different flavors. However, weak and the Yukawa interactions directly affected by the flavor of the participants to the interaction. When it comes to the beyond the standard model interactions there may be some new degrees of freedom that are affected by the flavors.\\
During a \textit{flavor changing} interaction flavor numbers change. There are two types of flavor changing interactions. If the interaction is between both up type and down type flavors or charged leptons and neutrinos, then it involves \textit{flavor changin charged current}(FCCC). For the interactions between either up type or down type flavors but not both and/or either charged leptons and neutrinos but not both, the \textit{flavor changing neutral currents} (FCNC) are involved. In the SM FCNC are suppressed at tree level because of flavor is conserved in the processes that involves neutral gauge bosons ($Z^0, g$ and $\gamma$) (See also appendix). Therefore, it makes FCNC are highly sensitive to the new physics.\par
\subsubsection{Weak interactions}
The electroweak interactions are summarized to the following form
\begin{eqnarray}
\mathcal{L}_{\mathrm{int}}^{\mathrm{EW}}=\mathcal{L}_{\mathrm{CC}}+\mathcal{L}_{\mathrm{NC}}
\end{eqnarray}
where $\mathcal{L}_{CC}$ and $\mathcal{L}_{NC}$ describe the charged current and neutral current interactions. In particular, the CC is given by \cite{Buras:1998raa}
\begin{eqnarray}
\mathcal{L}_{\mathrm{CC}}=\frac{g}{2 \sqrt{2}}\left(J_{\mu}^{+} W^{+\mu}+J_{\mu}^{-} W^{-\mu}\right)
\end{eqnarray}
where 
\begin{eqnarray}
J_{\mu}^{+}=J^1_{\mu}+iJ^2_{\mu}=\bar{U}_L\gamma_{\mu}D_L+\bar{l}\gamma_{\mu}\nu_{L}
\end{eqnarray}
and the NC is given by
\begin{eqnarray}
\mathcal{L}_{\mathrm{NC}}=-e J_{\mu}^{\mathrm{em}} A^{\mu}+\frac{g}{2 \cos \theta_W} J_{\mu}^{0} Z^{\mu}
\end{eqnarray}
where $e$ is the QED coupling. The neutral electromagnetic and weak currents are given by
\begin{eqnarray}
\begin{array}{c}{J_{\mu}^{\mathrm{em}}=\sum_{f} Q_{f} \overline{f} \gamma_{\mu} f} \\ {J_{\mu}^{0}=\sum_{f} \overline{f} \gamma_{\mu}\left(v_{f}-a_{f} \gamma_{5}\right) f}\end{array}
\end{eqnarray}
where 
\begin{eqnarray}
v_{f}=T_{3}^{f}-2 Q_{f} \sin ^{2} \theta_W, \quad a_{f}=T_{3}^{f}.
\end{eqnarray}
The $Q_f$ and $T_3^f$ denotes the charge and the third component of the weak isospin of left handed fermion. Important Feynman rules for the electro-weak interactions are given below. The photonic and gluonic vertices are vector like (V), the $W^{\pm}$ vertices involve only vector-axial vector $(V-A)$ like and the $Z^0$ vertices involve both $V-A$ and $V+A$ structures. Also, the vertices that involves higgs plays an important role. They are proportional to the top mass $m_t$. This relates to the CP violating decays and transitions.\par
In the figure \ref{fig:elemPropegators} we present Feynman rules for the propagators. 
\begin{figure}[H]
\centering
\includegraphics[width=8cm]{propegators.JPG}
\caption{\label{fig:elemPropegators}
Feynman rules for the propegators \cite{Buras:1998raa}.}
\end{figure}
The Feynman rules for the vertices are given in figure \ref{fig:elemVertices}
\begin{figure}[H]
\centering
\includegraphics[width=8cm]{elementaryvertices.JPG}
\caption{\label{fig:elemVertices}
Feynman rules for the vertices \cite{Buras:1998raa}.}
\end{figure}
\subsubsection{FCNC}
In the SM FCNC proceed only in loop induced interactions. In general these processes are classified as $\Delta F=1,2$ transitions, where $\Delta F$ represents the change in the flavor during the interaction. Until the 1970 there were only 3 observed quarks they are u,d and s. These three quarks mix with each other, and this mixing is governed by Cabibbo angle $\theta_c$. For this 3 quark system the weak neutra l current ($J_{N C}^{0}$) can be obtained as follows:
\begin{eqnarray}\label{3flavorNC}
J_{N C}^{0}=u \bar{u}+d^{\prime} \bar{d}^{\prime}+s^{\prime} \overline{s^{\prime}}=u \bar{u}+d \bar{d} \cos ^{2} \theta_{C}+s \bar{s} \sin ^{2} \theta_{C}+(s \bar{d}+\bar{s} d) \cos \theta_{C} \sin \theta_{C}
\end{eqnarray}    
where $d^{\prime}=d \cos \theta_{C}+s \sin \theta_{C}$. The last term in the equation (\ref{3flavorNC}) provides the direct flavor change neutral current for $s\rightarrow d$. In contrast, there was no experimental evidence for such direct flavor changing interaction. Historically, the first searches for the FCNC were carried out for the $K^+\rightarrow \pi^+e^+e$ and $K^+\rightarrow \pi^+\nu\bar{\nu}$. The upper limit for the branching rations for these searches were in the order of $10^{-6}$. This small branching ratio caused people to believe that FCNC are absent in the SM. In 1970 a new quantum number was introduced in the work of Glashow, Iliopoulos and Maiani (GIM) \cite{Glashow:1970gm}. This new quantum number is known as ``charmness". This GIM mechanism was introduced to cancel out the unobserved FCNC transition. The yet unobserved charm ($c$) quark completed the second quark family. The four quarks were arranged to two $SU(2)$ doublets.
\begin{eqnarray}
d^{\prime}=d \cos \theta_{C}+s \sin \theta_{C}
\end{eqnarray} 
\begin{eqnarray}\label{ccontribution}
s^{\prime}=-d \sin \theta_{C}+s \cos \theta_{C}.
\end{eqnarray}
The equation (\ref{ccontribution}) modifies the equation (\ref{3flavorNC}) such that the FCNC part cancel out. This explains the observation of no FCNC at leading order. However, box and penguin diagrams give rise to the small and rare FCNC interactions at one loop level, and they are experimentally measured. The GIM mechanism provided a quantitative prediction to the $c$ mass \cite{Gaillard:1974hs}. Later the charm quark was discovered by the SLACK and BNL. Because of this FCNC processes become very important beyond standard model (BSM) probes. In particular, $B$ meson FCNC played important role in development of SM.   As an example, the absence of large $B$ meson FCNC processes ruled out the possibilities of top less models that arisen in late 70's. These historical examples of how FCNC shape the SM motivates us studying it further.

\subsubsection{Penguin diagrams}
At one loop order the possible FCNC interactions are summerized by triple and quartic effective vertices. In literature these vertices are known as \textit{penguin} and \textit{box} diagrams. The name ``penguin" was coined by J.Ellis \cite{Ellis:1977uk}. Penguin diagrams are defined by single exchange of $W$ bosons. Whereas, the box diagrams contain two $W$ exchanges. \par
The importance of the penguin diagrams was pointed out in the work of Vainshtein, Zakharov, and Shifman \cite{Shifman:1975tn}. For instance, penguin diagrams are responsible for the enhancement of the $\Delta I=1/2$ amplitude compared to the $\Delta I=3/2$ amplitude in weak $K\rightarrow \pi\pi$ decays. The importance of the penguins to the CP violation was first pointed out by Bander, Silverman and Soni 
\cite{Bander:1979px}. They showed that the interference between the tree level diagrams and the penguin diagrams can give a large CP asymmetry in $B$ decays.\par
In $b$ transitions to lighter quarks such as $s$ and $d$, the penguin effects are rather pronounced. In these penguins $t$ quark is primarily contributing to the loop, and this provides a large coupling between $b$ and $t$ because $|V_{tb}|\sim 1$. This feature of $b$ penguins makes $b\rightarrow s$ and $b\rightarrow d$ transitions sensitive to $|V_{ts}|$ and $|V_{td}|$. \par
In particular, the decays such as $b\rightarrow s (d)$ are classified into a class of diagrams that are known as \textit{electromagnetic penguins}. In these decays a hard photon is emitted from a charged particle. This hard photon is an excellent experimental signature. The SM predicts  $\mathcal{B}(b\rightarrow s\gamma)=(3.36\pm 0.23)\times 10^{-4}$ \cite{Misiak:2015xwa}. This estimate is obtained by assuming a photon cut $E_{\gamma}>1.6\text{ GeV}$. The SM prediction is compared to the 2019 update of the 2018 particle data group (PDG) average $(3.49\pm 0.19)\times 10^{-4}$ \cite{Tanabashi:2018oca}.
\begin{figure}[H]
\centering
\includegraphics[width=8cm]{em_peng.JPG}
\caption{\label{fig:Btosd_fcnc}
The electromagnetic penguin diagram \cite{Lingel:1998fa}.}
\end{figure}

